\documentclass[10pt,a4paper]{book}

\begin{document}

\begin{flushleft}
  \textbf{128} \qquad  \tiny CHAPTER NINE
\end{flushleft}

\begin{flushleft}
  \textsl{\textbf{Setting the Research Question(s)}}
\end{flushleft}

Many imporant education questions lend themselves to consensus-building research techniques because the questions have no single, self-evident, or universal answer. A good question for the consensus technique draws out the participant's experiences and opinion. In particular, the question should be designed to draw out the particlipants'opinions on an important , but controversial , idea or their suggested solutions to a complex problem . Hopefully during the rounds of discussion, background detail or personal examples will be brought out and these will be used as incentives for the participants to converge on a consensus of opinion .

Like a survey,there is no single correct number of questions that makes a good consensus study;however , if there are too many questions,participants may become fatigued and abandon the process . This is especially important when participants are not engaged in real-time activities and they can leave the task without fear of hurting the researcher's feelings by an obvious virtual "walk out of the door".Remember also that the participants will see these same questions at least two times,and perhaps as many as five rimes,so we suggest not making the survey or the individual questions so long that participants will be reluctant to participate in subsequente rounds.Carefull timing with pilot subjects is usefull to be able to accurately inform participants of the estimated time required to complete the assigned tasks.

\begin{flushleft}
\textsl{\textbf{Selecting a Participant Sample}}
\end{flushleft}

Usually participants are purposively chosen because they have knowledge that is valuable in answering the research questions. The target audience for the research should be considered when selecting the participants and a sample selected that will be perceived as credible by these final arbitrators of the value of the research. Most of the techniques for focus groups (see Chapter 8) or survey samples (see Chapter 11) can be used to gather a sample for consensus-building study. Ethical issues related to privacy and confidentiality must also be addressed when soliciting participation (see Chapter 5). Invitations to participate canbe send by email (to save cost and time), but we suggest, in agreement with other researchers (e.g. Dillman,2000),that an initial letter on the letterhead of the sponsoring or affiliated organization is more likely to garner high rates of participation.

participant heterogeneity is also an issue in participant selection. Obviously, the participants should have enough in common to be able to knowledgeably discuss the research problem. if,however,the question is marked by deep polarization,it may be very difficult for the group to move to consensus. Alternatively, choosing only well-know proponents of a particular cause or issue involves bias and will likely not result in sufficient differentiation to stimultatc discussion or defense of different positions. Finally,if the rechnique chosen allows participants to know and interact directly with other members (as often accurs in online Nominal Group processes),large differences in status may impair honest and open discussion. The number of participants in most Delphi studies ranges from ten to thirty-fewer than ten participants may result in lack of ideas and problems with reliability, and more than thirty well-chosen participants

\end{document}
