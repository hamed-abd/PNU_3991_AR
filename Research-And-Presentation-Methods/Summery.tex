\documentclass [7pt]{beamer}
\usepackage{xcolor}	
\usepackage{tikz}
\usetheme{Frankfurt}
\useoutertheme{infolines}
\usepackage{ragged2e}
\begin{document}
\small
\section*{kholase safahat 127..129 }
\subsection*{hamed abdolrazagh }	
\begin{frame}
\justifying	
WIKI WEB system  Develop a research project focused on net manufacturing
\begin{flushleft}
\justifying
The WIKI WEB system is the latest emerging form of consensus building technique that is native to the network and designed by ward cunningham (https://www.c2 .com/cgi/wiki?Welcomevisitors). A WIKI site allows selected users or all users to jointly edit HTML documents. This means that any user can edit, delete, add, or change the page on which a particular group works. It maintains previous consensus, but is very informal and spontaneous and lacks the means to quantify differences between group members WIKIs have been adopted by various organizations, and this process has evolved with the development of different rules (such as limited membership) for different purposes. The first step in any research design is to plan the operational steps. For a consensus study, this plan revolves around setting up the research question (s), selecting a sample of participants, and deciding on the method (s) of interaction. Participants make decisions about technique analysis, and decisions about how to publish results.
\end{flushleft}
\end{frame}

\begin{frame}
Set up research questions (s)
\begin{flushleft}
\justifying
Many questions lend themselves to consensus research techniques because they do not have a single, obvious or definite answer. Draws a good question for the consensus of the participant's experiences and opinions, such as a poll. Single questions do not make for a good consensus study, but if there are too many questions, participants get tired and leave the process. Accurate scheduling with test subjects is helpful to accurately inform participants of the estimated time required to complete the task. Participants are usually purposefully selected to have sufficient knowledge to answer research questions. Ethical issues related to privacy and confidentiality should also be considered when applying for participation. Company invitations can be emailed (to save cost and time). Obviously, participants must have enough in common to be able to discuss research. The number of participants in most Delphi studies is from ten to thirty.
\end{flushleft}
\end{frame}

\begin{frame}
\justifying
New information is rarely obtained from relatively homogeneous groups. In general, the panel of experts should be represented by experience and expertise in various fields, and the heterogeneity of participants should be maintained to ensure the accuracy of the results. Delphi studies are known for their significant time commitments. To prevent people who agree to participate from attending, the researcher should try to draw a good line between allowing a wide range of information to be shared and summarizing each round. In particular, the researcher should conduct the research in such a way that it does not lead to a large amount of information that would take the time of the participants when reviewing the summaries. On the other hand, the researcher must use his personal discretion to decide what is in the summaries. To be ethical and maintain credibility, the researcher must state who and what part of the partnership is involved. Finally, the reliability of group responses is related to the size of the group and increases with it. However, statistically, one should not pay too much attention to the size of the group, because inferential statistics are not used. Topics related to web formatting or email surveys are given in Chapter I1.
\end{frame}
\end{document}